\documentclass{article}
\usepackage{amsmath}
\begin{document}
Hello World!

This is where I type my main text.

% quotation marks are shit
`single quotes'
``double quotes''

% special characters -> %, #, &, $. Escape them for usage in text

This is some other text with \% and \#.

a pair of \$ signs enclose mathematical text
$y=mx+n$

(only used in math mode?) \^  for superscripts \_ for subscripts and \{ \} for grouping stuff

$y = c_2^2 + c_{234}^{65}$

if it's big, use the equation context
\begin{equation}
-\frac{1}{12} = \sum_{i=0}^{+\infty}i
\end{equation}

\$\dots\$ is equivalent to \\begin\{math\} .. \\end\{math\}

and the equation context is equivalent to \$\$ \dots  \$\$

the environments are used to create many different environments:
-> Itemize
-> Enumerate

Packages are libraries of extra commands and environemnts to use.
Lots of free environments to use.

we have to load the package in the preamble (between the definition of the document class and
the document environment)

asmath stuff
unnumbered equations: equation-star environment
\begin{equation*}
    \Omega = \infty
\end{equation*}


Latex trates adjacent letters as variables multipleid together, which is not always the case
as with operators (min, max, etc)

\begin{equation*}
    min_{x,y} = bad
\end{equation*}

\begin{equation*}
    \min_{x,y} = good
\end{equation*}

you can also define operators
\begin{equation*}
    \operatorname{Cov}(R_i, R_m)
    \operatorname{Var}(R_m)
\end{equation*}


Align a set of equations:
\begin{align*}
    (x + 1)^3   &= (x+1)(x+1)(x+1 ) \\
                &= (x+1)(x^2 + 2x + 1) \\
                &= x^3 + 3x^2 + 3x + 1
\end{align*}


\end{document}
